\documentclass[a4paper,11pt,fleqn]{article}
\usepackage[utf8]{inputenc}
%\usepackage[italian]{babel}
%\usepackage{lipsum}
\usepackage{graphicx}
\usepackage[font=small,labelfont=bf]{caption}
\usepackage{multirow}
\usepackage{hyphenat}
\usepackage{sectsty}
\usepackage{amsmath}
\usepackage{amssymb}
%\usepackage{subfigure}
%\usepackage{color}
\usepackage{hyperref}


\title{Homework 2}
\author{Alessandro Bianchetti}

\begin{document}
\maketitle

\section{Exercise 4}
The purpose here is to prove that the characteristic function of a MVN is given by
\begin{equation}
    \phi(\bf{k}) = \exp{(-i\mu^T\cdot \bf{k}-\frac{1}{2}\bf{k}^T C^{-1}\bf{k})}
\end{equation}
Let's first remind the definition of characteristic function:
\begin{equation}
    \label{eqn:char}
    \phi(\bf{k}) = \int d^nx \exp{(-i\bf{k}\cdot \bf{x} )p(\bf{x})}
\end{equation}
Also recall the form of the MVN function
\begin{equation}
    \mathcal{N}(\bf{x}|\bf{\mu}, C) = \frac{1}{(2\pi)^{N/2}\sqrt{\det{C}}} \exp{\left\{-\frac{1}{2}(\bf{x}-\bf{\mu})^T C^{-1} (\bf{x}-\bf{\mu})\right\}}
\end{equation}
So let's plug this definition inside \ref*{eqn:char} and see what happens. First,
we're going to apply the square completion technique, thus solving the Fourier Transform.
Then we're going to do it by rotating the covariance matrix C making it diagonal.

\subsection{Method 1: square completion}
For simplicity, from now on we refer to the normalization factor of the multivariate Gaussian as $A$.
\begin{gather*}
    \phi(\bf{k}) = \int d^nx \exp{(-i\bf{k}\cdot \bf{x}) A \exp{\left\{-\frac{1}{2}(\bf{x}-\bf{\mu})^T C^{-1} (\bf{x}-\bf{\mu})\right\}}} = \\
    = \int d^n u A \exp{\left\{ -\frac{1}{2}\bf{u}^TC^{-1}\bf{u} - i\bf{k}^T(\bf{u}+\bf{\mu}) \right\}}
\end{gather*}
where we replaced $\bf{u} = \bf{x}-\bf{\mu}$. Let's play on the exponent a bit in order to build a new squared
quantity at the exponent, this time involving the frequency $\bf{k}$.
\begin{multline*}
    -\frac{1}{2}\bf{u}^T C^{-1}\bf{u} - i\bf{k} \cdot (\bf{u}+\bf{\mu}) = -\frac{1}{2}\bf{u}^TC^{-1}\bf{u} - i \bf{k}\cdot \bf{u} -i \bf{k}\cdot \bf{\mu} = \\
    = -\frac{1}{2}\bf{u}^TC^{-1}\bf{u} - \frac{i}{2} \bf{k}\cdot \bf{u} - \frac{i}{2} \bf{k}\cdot \bf{u} -i \bf{k}\cdot \bf{\mu} = \\
    = -\frac{1}{2}\bf{u}^TC^{-1}\bf{u} - \frac{i}{2} \bf{k}^T \bf{u} - \frac{i}{2} \bf{u}^T \bf{k} -i \bf{k}\cdot \bf{\mu} = \\
    = -\frac{1}{2}\bf{u}^TC^{-1}\bf{u} - \frac{i}{2} \bf{k}^T C C^{-1} \bf{u} - \frac{i}{2} \bf{u}^T C^{-1}C \bf{k} -i \bf{k}\cdot \bf{\mu} = \\
    = = -\frac{1}{2}\bf{u}^TC^{-1}\bf{u} - \frac{i}{2} \bf{k}^T C C^{-1} \bf{u} - \frac{i}{2} \bf{u}^T C^{-1}C \bf{k} +\frac{1}{2}\bf{k}^T C \bf{k} -\frac{1}{2}\bf{k}^T C \bf{k} -i \bf{k}\cdot \bf{\mu} = \\
    = -\frac{1}{2}(\bf{u}+iC\bf{k})^T C^{-1} (\bf{u}+iC\bf{k}) -\frac{1}{2}\bf{k}^T C \bf{k} - i \bf{k} \cdot\bf{\mu}  
\end{multline*}
Therefore, now 
\begin{gather*}
    \phi(\bf{k}) = \exp{\left[-\frac{1}{2}\bf{k}^T C \bf{k} - i \bf{k}\cdot \bf{\mu}\right]} \int d^n u A \exp{\left[ -\frac{1}{2}(\bf{u}+iC\bf{k})^T C^{-1} (\bf{u}+iC\bf{k}) \right]}
\end{gather*}
and we can again change variable $\bf{t} = \bf{u}+iC\bf{k}$, so that
\begin{gather*}
    \phi(\bf{k}) = \exp{\left[-\frac{1}{2}\bf{k}^T C \bf{k} - i \bf{k}\cdot \bf{\mu}\right]} \int d^n t A \exp{\left[ -\frac{1}{2}\bf{t}^T C^{-1} \bf{t} \right]}
\end{gather*}
but in this last equation the integral is a perfectly normalised Gaussian integrated over infinity, meaning
it corresponds to 1. Finally:
\begin{gather*}
    \phi(\bf{k}) = \exp{\left[-\frac{1}{2}\bf{k}^T C \bf{k} - i \bf{k} \cdot \bf{\mu}\right]}
\end{gather*}


\paragraph{Alternative solution}
It is perhaps possible to try out a different path. In fact, after the first substitution (introducing $\bf{u}$), we can immediately
make the following change of variables, switching to $\bf{t}$:
\[
    \sqrt{C}\bf{t} = \bf{u}
\]
where $\sqrt{C}$ is defined as the squared root of the matrix C, meaning a matrix such that $\sqrt{C} \sqrt{C} = C$. This means that
\[
\bf{u}^T C \bf{u} = (\sqrt{C}\bf{t})^T C^{-1}\sqrt{C}\bf{t} = \bf{t} \cdot \bf{t} = t^2    
\]
Such a change of coordinates will also introduce the determinant of the transformation matrix $\sqrt{C}$ in the integral. Moreover,
$\det{\sqrt{C}}=\sqrt{\det{C}}$ due to Binet's Theorem. 
\begin{gather*}
    \phi(\bf{k})= \int d^n u A \exp{\left\{ -\frac{1}{2}\bf{u}^TC^{-1}\bf{u} - i\bf{k}^T(\bf{u}+\bf{\mu}) \right\}} = \\
    = A\sqrt{\det{C}}\int d^n t e^{-i\bf{k}\cdot \bf{\mu}}e^{-i\bf{k}\sqrt{C}\bf{t}}e^{-\frac{1}{2}t^2} = A\sqrt{\det{C}}e^{-i\bf{k}\cdot \bf{\mu}} \int d^n t e^{-\frac{1}{2}[t^2+2i\bf{k}\sqrt{C}\bf{t}]}
\end{gather*}
Here we perform a square completion by adding and subtracting the term $\bf{k}^TC\bf{k}$
\begin{gather*}
    \phi(\bf{k}) = A\sqrt{\det{C}} e^{-i\bf{k}\cdot \bf{\mu}}e^{-\frac{1}{2}\bf{k}^T C \bf{k}}\int d^nt e^{-\frac{1}{2}[\bf{t}+i\sqrt{C}\bf{k}]^2}
\end{gather*}
A final substitution $\bf{y}=\bf{t}+i\sqrt{C}\bf{k}$ leads to 
\begin{gather*}
    \phi(\bf{k}) = A\sqrt{\det{C}} e^{-i\bf{k}\cdot \bf{\mu}}e^{-\frac{1}{2}\bf{k}^T C \bf{k}}\int d^ny e^{-\frac{1}{2}y^2}
\end{gather*}
But the integral in the latter is equal to $(2\pi)^{N/2}$, and therefore it cancels out with the term
$A\sqrt{\det{C}}=(2\pi)^{-N/2}$. So we're left with 
\begin{gather*}
    \phi(\bf{k}) = e^{-i\bf{k}\cdot \bf{\mu}}e^{-\frac{1}{2}\bf{k}^T C \bf{k}}
\end{gather*}
which is again the expected result.

\subsection{Method 2: diagonalization}


\end{document}
